%% LyX 2.2.2 created this file.  For more info, see http://www.lyx.org/.
%% Do not edit unless you really know what you are doing.
\documentclass[aps,pra,graphicx]{revtex4-1}
\setcounter{secnumdepth}{3}
\usepackage{amsmath}
\usepackage{graphicx}

\makeatletter

%%%%%%%%%%%%%%%%%%%%%%%%%%%%%% LyX specific LaTeX commands.
%% Because html converters don't know tabularnewline
\providecommand{\tabularnewline}{\\}

%%%%%%%%%%%%%%%%%%%%%%%%%%%%%% User specified LaTeX commands.
%% ****** Start of file aiptemplate.tex ****** %
%%
%%   This file is part of the files in the distribution of AIP substyles for REVTeX4.
%%   Version 4.1 of 9 October 2009.
%%
%
% This is a template for producing documents for use with 
% the REVTEX 4.1 document class and the AIP substyles.
% 
% Copy this file to another name and then work on that file.
% That way, you always have this original template file to use.

%\documentclass[aip,reprint]{revtex4-1}

\makeatother

\begin{document}

\title{Fluctuating coherence in approximate ground states of a tightly confined
Bose gas}

\author{Rodney E. S. Polkinghorne}
\email[]{rpolkinghorne@swin.edu.au}


\author{Bogdan Opanchuk}
\email[]{bopanchuk@swin.edu.au}


\author{Peter D. Drummond}
\email[]{pdrummond@swin.edu.au}


\affiliation{Centre for Quantum and Optical Science, Swinburne University of Technology}

\date{\today}
\begin{abstract}
It is often supposed that the ground state of a tightly confined Bose
gas is a coherent state. This is true in the limit of very tight confinement,
where the extent of the ground state is small compared to the healing
length of the gas. However, there is an intermediate regime, where
the extent of the gas is comparable to its healing length. In this
regime, the positions of the bosons are correlated. In an Bogoliubov
approximation, only a few sound wave modes have states significantly
different from coherent states; this would cause a gas in a coherent
state to undergo oscillations in its coherence function.
\end{abstract}
\maketitle

\section{Introduction}

When the first Bose-Einstein condensates were made in dilute gases
\cite{sci-269-198,prl-75-3969}, research focused on their global
order parameter \textbf{\cite{sci-275-637}}. If the condensed gas
has many bosons, its order parameter is a classical field, which diffracts
and interferes much as light does. A condensate of non-interacting
bosons occupies a single orbital, and is exactly analogous to coherent
light. When the bosons interact, the condensate occupies multiple
orbitals, but it can often be approximated by a classical field with
a nonlinear equation of motion \textbf{citation for GPE}.

There is a limited amount to be said about interference in a totally
coherent field, and attention shifted to other aspects of BECs. More
recently, it has become possible to form the gases into very elongated
clouds \cite{prl-79-549}, where the atomic field is effectively a
plane or a line. The density of states in these low-dimensional geometries
differs from that in three dimensions, with the consequence that a
condensate with a global order parameter does not exist in a one-dimensional
Bose gas at any finite temperature \cite{prl-85-3745}. This raised
a new set of questions about spatial coherence in these systems, questions
which interference experiments are the obvious means to answer. Such
experiments have been underway in several laboratories for some years
\cite{nap-1-57,nat-462-74}.

As well as the experiments, theoretical and numerical studies have
explored the spatial coherence of the Bose gas in one-dimension. This
is a many-particle quantum system, and the methods used to calculate
its properties usually describe the gas in second quantisation. These
simulations must use some approximate forms for the state of the interacting
condensate. The coherent state is the limit of non-interacting bosons,
and the Hamiltonian can expanded as a second order Taylor series around
this limit. The Bogoliubov state is the ground state of that approximate
Hamiltonian, and it is often used in theoretical work.

The exact ground state of an interacting Bose gas is known in first
quantisation, but it is difficult to compare this to the approximate
second-quantised ground states. As a result, it is hard to estimate
the uncertainty introduced by the approximate ground state.

This paper will test the approximate ground states by computing their
dynamics. The exact ground state is stationary, and so the quality
of the approximation can be assessed by how nearly stationary the
approximate ground states are. We will find that even a coherent state
has a stationary density, because repulsion puts a tight constraint
on how much the density can vary. However, the correlations in the
positions of the bosons are a more sensitive test. These can be quantified
by the second order coherence function, which can be measured by counting
coincident bosons. The main results of this paper are the dynamics
of this coherence function, which indicate the quality of the approximate
ground states for a range of gas sizes and repulsion parameters.

Independent approximation, TW dynamics.

\section{The interacting Bose gas and its ground state}

In SI units, the Hamiltonian of an interacting but untrapped Bose
gas is

\begin{equation}
\hat{H}_{\text{SI}}=\frac{\hbar^{2}}{2m}\hat{H}=\frac{\hbar^{2}}{2m}\int\hat{\psi}{}^{\dagger}(x)\left(-\nabla^{2}-\xi^{-2}\right)\hat{\psi}(x)+4\pi a\left(\hat{\psi}^{\dagger}(x)\right)^{2}\hat{\psi}^{2}(x)\,dx.\label{eq:i}
\end{equation}
The bosons are identical particles of mass $m$, that interact through
a contact potential with scattering length $a$, and are annihilated
by the second-quantised field operator $\hat{\psi}(x)$. Each boson
has been assigned a potential energy $-\hbar^{2}/\left(2m\xi^{2}\right)$,
enabling some of the following calculations to be simplified by allowing
the chemical potential of the interacting ground state to be adjusted
to zero. This paper uses scaled units with $\hbar^{2}=2m$, which
reduce the Hamiltonian from $\hat{H}_{\text{SI}}$ to $\hat{H}$.

It is well known that a Bose gas condenses below a critical temperature,
a phase transition for which the field of the bosons occupying the
condensate orbital is an order parameter. The state of these condensed
bosons is often taken to be a coherent state $\left|\psi(x)\right\rangle $,
in which case the coherent amplitude $\psi(x)$ is an order parameter.
If a variational ansatz is imposed, under which the state of the Bose
field remains coherent but the order parameter may vary over time,
this order parameter has a variational solution $\psi(t,x)$ such
that the coherent ansatz$\left|\psi(t,x)\right\rangle $ satisfies
$\frac{\partial}{\partial t}\left|\psi(t,x)\right\rangle =-i\hat{H}\left|\psi(t,x)\right\rangle $
in the Dirac-Frenkel sense \textbf{citation}. This solution satisfies
the Gross-Pitaevskii equation,
\begin{equation}
\frac{\partial\psi}{\partial t}=-i\left(-\nabla^{2}-\xi^{-2}+8\pi a\left|\psi\right|^{2}\right)\psi.\label{eq:vii}
\end{equation}
The assumption of a coherent state reduces the quantum Bose field
to a classical order parameter field, which is far easier to reason
about and calculate with.

The quantity $\xi=\left(8\pi a\left|\psi\right|^{2}\right)^{-1/2}$
is called the healing length. This paper will consider uniform systems,
in which the healing length is constant, and the constant $\xi$ in
Equation \ref{eq:i} will always be set equal to it. In the classical
field theory of the order parameter, the healing length is the only
physically meaningful property of the gas itself. It is the scale
over which the order parameter drops to zero where the gas is confined
at a hard edge; this is the shortest scale on which features can occur
in the order parameter. The healing length sets the boundary between
long distances on which the density of the gas can vary, and short
ones on which repulsion makes it uniform. As well as setting $\hbar^{2}=2m$,
this paper will use the healing length as its length unit, at which
point all mechanical properties of the bosons scale to dimensionless
quantities.

In the classical field governed by Equation \ref{eq:vii}, all parameters
except the healing length are properties of the apparatus that holds
the gas, not of the gas itself. This paper will focus on uniform gases
with periodic boundary conditions, whose only other parameter is the
size of the gas. The transverse area that will be discussed shortly
is another of these geometric parameters.\textbf{ Explain what it
means for the healing length to be smaller than the system size with
periodic boundary conditions and a uniform condensate. How physical
is this?}

Suppose that the gas is held in a cigar-shaped trap, which confines
it tightly along the $z$-axis. The bosons will only occupy the ground
orbital of the transverse potential, at least in the limit of very
tight confinement, and their field operator will factorise in the
form $\hat{\psi}(x,y,z)=\theta(x,y)\hat{\psi}(z)$. This reduces the
Hamiltonian to
\begin{equation}
\hat{H}=\int\hat{\psi}{}^{\dagger}(z)\left(-\partial_{z}^{2}-\xi^{-2}\right)\hat{\psi}(z)+\frac{4\pi a}{A}\left(\hat{\psi}^{\dagger}(z)\right)^{2}\hat{\psi}^{2}(z)\,dz,\label{eq:v}
\end{equation}
where $A=1/\int\left|\theta\right|^{4}$ is the area occupied by the
transverse ground orbital $\theta$. In the case of a uniform density
$n$, it is possible to define a dimensionless parameter $\gamma$
such that the repulsion coefficient is $4\pi a/A=n\gamma$ and the
healing length is $\xi=1/\sqrt{2\gamma}n$. This is called the Lieb-Liniger
parameter, and it measures how far the gas lies from the coherent
limit of non-interacting bosons such as photons. When lengths are
scaled to the healing length, the Hamiltonian becomes

\begin{equation}
\hat{H}=\int\hat{\psi}{}^{\dagger}(z)\left(-\partial_{z}^{2}-1\right)\hat{\psi}(z)+\sqrt{\frac{\gamma}{2}}\left(\hat{\psi}^{\dagger}(z)\right)^{2}\hat{\psi}^{2}(z)\,dz,\label{eq:ii}
\end{equation}
and the boson density is $\left\langle \hat{\psi}{}^{\dagger}(z)\hat{\psi}(z)\right\rangle =\xi n=1/\sqrt{2\gamma}$.

\rule[0.5ex]{1\columnwidth}{1pt}

The ground state of this one-dimensional Bose gas was found by Lieb
and Liniger \cite{prx-130-1605}. It has a fairly simple form in first
quantisation: the many-particle wave function is composed from pieces
of free-particle wave functions, with appropriate boundary conditions
on the hyperplanes where two of the particles coincide. This form
is very complicated in second quantisation, so some approximations
are often used. These are described in Section \ref{sec:Coherent}.

To date, many experiments have studied the density of a Bose gas,
but few have studied higher order coherence functions. These gases
are observed by taking photographs, and the density is the observable
that this naturally measures. More recently, one-dimensional BEC interferometers
have become available \cite{nap-1-57,nat-462-74}, which allow higher
order coherence function to be measured. This paper will focus on
the second order coherence function 
\begin{equation}
g^{2}(x)=\frac{\left\langle \hat{\psi}^{\dagger}(x')\hat{\psi}^{\dagger}(x'+x)\hat{\psi}(x'+x)\hat{\psi}(x')\right\rangle }{\left\langle \hat{\psi}^{\dagger}(x')\hat{\psi}(x')\right\rangle \left\langle \hat{\psi}^{\dagger}(x'+x)\hat{\psi}(x'+x)\right\rangle }\label{eq:iv}
\end{equation}
that was introduced by Glauber \cite{prx-130-2529}. This is independent
of $x'$ in the usual case of a gas that is uniform along the long
axis of the trap. Note that, according to Equation \ref{eq:ii}, $g^{(2)}(0)=\sqrt{2/\gamma}\left\langle V\right\rangle $,
where $V$ is the density of repulsion energy.

\section{Coherent and Bogoliubov approximations\label{sec:Coherent}}

In the regime of low interactions, where there are many atoms per
healing length and $\gamma$ is small, the second term in Equation
\ref{eq:ii} is much smaller than the first term. In this limit, it
makes sense to approximate the ground state of the interacting gas
with that of a gas of free bosons. This can taken to be a multimode
coherent state $\left|\psi_{0}\right\rangle $, an eigenstate of the
field annihilation operator such that $\hat{\psi}(x)\left|\psi_{0}\right\rangle =\psi_{0}(x)\left|\psi_{0}\right\rangle $
\cite{prx-131-2766}. The wave function $\psi_{0}$ is an order parameter
for the boson field. In general, this satisfies a Gross-Pitaevskii
equation \textbf{citation}, but in a uniform gas it is constant with
$\psi_{0}(x)=\sqrt{n}$.

There are some objections in principle to a coherent ground state.
The major one is that it superposes states with different numbers
of particles, and that the ground state should actually be a state
with fixed total number of the form $\left|N\right\rangle =\left(\int\psi_{0}(x)\hat{\psi}^{\dagger}(x)\,dx\right)^{N}\left|0\right\rangle $
\textbf{citation}. However, it is here being considered as an approximation,
and there are reasons to think that observable properties predicted
by it will agree with those seen in experiments. These are based in
the fact that certain mixtures of number states are identical to certain
mixtures of coherent states. Precisely, when $P(N)$ is a Poisson
distribution, with mean $\bar{N}=\int\left|\psi_{0}\right|^{2}$,
then
\begin{equation}
\sum_{n=0}^{\infty}P(N)\left|N\right\rangle \left\langle N\right|=\text{\ensuremath{\frac{1}{2\pi}}}\int_{0}^{2\pi}\left|e^{i\phi}\psi_{0}\right\rangle \left\langle e^{i\phi}\psi_{0}\right|\,d\phi.\label{eq:iii}
\end{equation}
On the one hand, the global phase $\phi$ is not observable, and the
value of any observable in this mixed state is the same as in the
coherent state $\left|\psi_{0}\right\rangle $ . On the other hand,
provided that the number of particles in the gas is large, the value
of an observable in the number state $\left|\bar{N}\right\rangle $
should be very close to its average over the Poission distribution
$P(N)$. So observables will take the same expectation values in the
coherent state and in the number state. This is especially true for
observables that can be measured repeatably in experiments, because
the number of particles in a condensed gas can typically not be reproduced
with better than a Poisson distribution \textbf{citation}. There are
exceptions. For instance, in some of the experiments currently being
done by Greiner, the parity of the total number of particles in the
gas is measured. A coherent state would not be an adequate approximation
when measurements such as these are being contemplated, but it is
usually obvious how to modify it.

\textbf{That argument is valid for a coherent state, but it is equally
valid for any other state with small number variance. Presumably we
use coherent states instead of squeezed or cat states because they
are simpler.}

In the regime where $\gamma$ is less than 1, but not much less, the
approximation of a coherent ground state can be extended in a way
due to Bogoliubov \textbf{citation}. The atomic field is assumed to
have the form $\hat{\psi}(x)=\psi_{0}(x)+\hat{\delta\psi}(x)$, where
the operator $\hat{\delta\psi}$ is small compared to the order parameter
$\psi_{0}$. The Hamiltonian of Equation \ref{eq:ii} is expanded
as a binomial in $\psi_{0}$ and $\hat{\delta\psi}$, and the terms
of third and forth order in $\hat{\delta\psi}$ are dropped. For a
uniform gas of density $\psi_{0}^{2}=1/\sqrt{2\gamma}$, this results
in a Hamiltonian
\[
\hat{H}(x)=-\frac{1}{2\sqrt{2\gamma}}+\left(-\partial_{z}^{2}+1\right)\delta\hat{\psi}^{\dagger}\delta\hat{\psi}+{\textstyle \frac{1}{2}}\delta\hat{\psi}^{2}+{\textstyle \frac{1}{2}}\left(\delta\hat{\psi}^{\dagger}\right)^{2}.
\]

A Bogoliubov transform can be performed, in which $\delta\hat{\psi}$
is expanded as
\[
\delta\hat{\psi}(x)=\sum_{j=1}^{\infty}\left(u_{j}^{\ast}\hat{a}_{j}-v_{j}\hat{a}_{j}^{\dagger}\right)w_{j}(x).
\]
The orthonormal modes $w_{j}$ are given by
\[
w_{2n-1}(x)={\textstyle \sqrt{\frac{2\xi}{L}}}\sin(\kappa_{n}x),\qquad w_{2n}(x)={\textstyle \sqrt{\frac{2\xi}{L}}}\cos(\kappa_{n}x),
\]
where $L$ is the length of the one-dimensional gas in SI units, and
the modes have scaled wave numbers $\kappa_{n}=2\pi n\xi/L$ . The
mode operators $\hat{a}_{j}$ and $\hat{a}_{j}^{\dagger}$ should
satisfy the Bose commutation relation $\left[\hat{a}_{j},\hat{a}_{j}^{\dagger}\right]=1$,
which requires the coefficients $u_{j}$ and $v_{j}$ to satisfy $u_{j}^{2}-v_{j}^{2}=1.$
If the coefficients are chosen such that
\[
u_{2n-1}^{2}=u_{2n}^{2}=\frac{\kappa_{n}+\kappa_{n}^{-1}}{2\sqrt{\kappa_{n}^{2}+2}}+\frac{1}{2}\qquad v_{2n-1}^{2}=v_{2n}^{2}=\frac{\kappa_{n}+\kappa_{n}^{-1}}{2\sqrt{\kappa_{n}^{2}+2}}-\frac{1}{2},
\]
there are no interactions between the modes, and the Hamiltonian takes
the diagonal form

\[
H_{B}=\text{const.}+\sum_{n}\kappa_{n}\sqrt{\kappa_{n}^{2}+2}\left(a_{n}^{\dagger}a_{n}+b_{n}^{\dagger}b_{n}\right).
\]

The frequency of the quasiparticle mode with wavelength $\kappa_{n}$
is $\omega_{n}=\kappa_{n}\sqrt{\kappa_{n}^{2}+2}$. In the limit of
short wavelength with $k\gg\xi$, this asymtotes to a quadratic dispersion
relation $\omega\approx1+\kappa^{2}$, and the quasiparticle energy
is just the chemical potential plus the kinetic energy of an atom
with the same wavelength. In the long wavelength limit, with $k\ll\xi$,
the dispersion relation becomes linear, with $\omega_{n}\approx\sqrt{2}\kappa_{n}$.
Long wavelength excitations act like sound waves, and in the units
used here, the Bogoliubov speed of sound is $\sqrt{2}$.

\includegraphics{/Users/rpolkinghorne/Desktop/thesis/figures/uv-1}

\section{Dynamics of a coherent initial state}

Suppose that a Bose gas is prepared in a coherent state. One orbital
is special, because it has the same form as the order parameter. The
state of this orbital is coherent, with nonzero amplitude; it does
not correspond to any mode in the Bogoliubov description. The state
of the noncondensate orbitals is illustrated in Figure \ref{fig:i}(a).
On the left hand side, at $t=0$, the atomic orbital is not occupied,
and its state is a vacuum. The corresponding sound wave must be in
a squeezed vacuum state. The dynamics are shown in the figure. As
the state of the orbital alternates between a vacuum and a squeezed
vacuum, its occupation oscillates sinusoidally. Figure \ref{fig:i}(b)
shows that this is the case in the simulations. The Fourier components
of the orbital occupations are plotted, and they form a sharp Bogoliubov
dispersion curve. (In the wrong units for this paper, which Rodney
will fix after he plots the figures he needs for his thesis.)

The orbitals are kinetic energy eigenstates, so the kinetic energy
of the gas is simply the sum of their energies weighted by occupation.
Figure \ref{fig:i}(c) shows the terms in this sum. The kinetic energy
will be the sum of a small number of discrete sinusoidal functions.

Physically, the sum of the kinetic energy and the potential energy
is conserved. However, as noted above, the potential energy is proportional
to $g^{(2)}(0)$. Therefore, to conserve energy, the coherence function
must oscillate in the opposite sense to the kinetic energy. In particular,
it will have the same frequency components.

\begin{figure}

\caption{The vacuum state of an atomic orbital is a squeezed vacuum state of
the corresponding Bogoliubov sound wave mode. Under the Bogoliubov
hamiltonian, the sound wave amplitude merely rotates in phase space.
The equivalent atomic field periodically squeezes and returns to a
vacuum state, so the number of atoms occupying the orbital oscillates
at the sound wave frequency. The numerical results are for a truncated
Wigner simulation with parameters $\gamma=0.1$, axis length 10.\label{fig:i}}
\includegraphics{cobo-1}\includegraphics{resp080817a}
\end{figure}


\section{Oscillating corellations}

One test of whether a system is in its ground state is that the ground
state is stationary. This provides a way to investigate how close
the coherent and Bogoliubov states are to the actual ground state
of a Bose gas. If the dynamics of a gas can be computed, when it has
one of these as its initial state, then the observables should have
constant values for the actual ground state, but their values are
likely to vary with time for other states.

The most obvious observable is the density of gas. However, this is
not likely to vary in any reasonable approximation to the ground state.
All of these states are derived from an equilibrium order parameter
that satisfies the Gross-Pitaevskii equation with minimum energy.
Any state that approximates the ground state will lack enough kinetic
energy to supply the repulsion energy required for the density to
change very much from this equilibrium value.

On the other hand, the correlation and coherence 

A coherent state in the Bogoliubov approximation, squeezing, rotation,
large and tight limits, oscillations in the intermediate regime.

In the short length limit, the gas becomes a single mode Bose gas,
and its dynamics are equivalent to a quartic oscillator with Hamiltonian
$H=\text{constant}\times\left(a^{\dagger}\right)^{2}a^{2}$. A coherent
state is not an eigenstate of this Hamiltonian, and, given an initially
coherent state, the coherence functions will not factorise. In particular,
$\left\langle a\right\rangle $ decays to zero over time, but the
Hamiltonian conserves the atom number $\left\langle a^{\dagger}a\right\rangle $.
However, a coherent initial state does remain coherent to second order,
with $g^{2}(0)=1$. The Bogoliubov approximation fails in this limit:
the approximate ground state is a coherent state.

\section{The single orbital limit}

In a very tight trap, there is not enough kinetic energy to get bosons
out of their ground orbital. Therefore the field reduces to a quartic
oscillator, \textbf{under the following approximations.}

It is interesting to note that the second order coherence $g^{2}$
is conserved in a quartic oscillator.

\textbf{How does the Lieb-Liniger ground state reduce to a number
state in this limit?}

In this limit, where $k\xi\gg1$ for every orbital, the non-condensate
orbitals of the Bogoliubov state reduce to vacuum states.

\section{The truncated Wigner approximation}

Although the Bogoliubov ground state is only an approximate eigenstate
of the atomic field hamiltonian, it is an exact eigenstate of the
second-order approximation to the hamiltonian that formed the basis
for the discussion in the last section. In order to investigate how
stationary the Bogoliubov state is under the exact hamiltonian, it
is necessary to solve for the dynamics governed by that hamiltonian.
It is not practical to do so exactly, as with almost all problems
in quantum field dynamics, but it is possible to solve for the dynamics
using approximations that are independent of the Bogoliubov approximation.

The simplest way to do this is the truncated Wigner method \cite{jcp-65-1289,epl-21-279,pra-58-4824}.
This method represents the quantum state of a system as a field over
the classical phase space of that system, by means of the Wigner transform
\cite{prx-40-749}. The field is treated as a probability density
for classical states, which evolves according to the classical Hamiltonian;
then the Weyl transform converts the time-evolving field back to a
density operator, which is treated as the time-dependent state of
the system. There are arguments that the classical dynamics should
approximate quantum dynamics, for a second quantised field, when the
field is expanded over modes such that there are many excitations
in each mode \cite{aip-57-363}.

This is very simple to put into practice as a Monte-Carlo method,
where an ensemble of classical configurations is drawn from the Wigner
distribution, and these sample configurations are propagated under
the classical laws of motion. In the calculations reported here, each
configuration is an order parameter for the atomic field, and the
classical law of motion is the Gross-Pitaevskii equation.

\section{Truncated Wigner simulations}

\begin{figure}

\caption{1D gas with $\gamma=1$, length 10, coherent state. Wave number truncated
to make kinetic energy converge. Something still not right.}
\includegraphics{resp080817b}\includegraphics{resp080817c}\includegraphics{resp080817d}

\end{figure}

\begin{figure}
\caption{Quadratures}
\begin{tabular}{|c|c|}
\hline 
\includegraphics[width=0.45\textwidth]{resp090617e} & \includegraphics[width=0.45\textwidth]{resp090617f}\tabularnewline
\hline 
 & \tabularnewline
\hline 
 & \tabularnewline
\hline 
 & \tabularnewline
\hline 
\end{tabular}

\end{figure}
There is a limit where a coherent state is stationary, and a broader
limit where a Bogoliubov state is stationary but a coherent state
oscillates. Note that the oscillations of the coherent state seem
to converge to a fixed pattern in the small $\gamma$ limit, although
the timescale on which the mean quadratures decay becomes slower.
In the conditions simulated here, with $\gamma\le0.1$, the value
of $g^{2}$ in the exact Lieb-Liniger ground state agrees with the
value in the Bogoliubov state to within a few percent \cite{prl-91-040403},
so the Bogoliubov graphs of $g^{2}$ are exact to within a few percent
at $t=0$.
\begin{figure}
\caption{Second order coherence of a 1D Bose gas starting from coherent and
Bogoliubov states, for various Leib-Liniger parameters and sizes of
gas}
\begin{tabular}{|c|c|}
\hline 
\includegraphics[width=0.45\textwidth]{resp090617g} & \includegraphics[width=0.45\textwidth]{resp090617h}\tabularnewline
\hline 
\includegraphics[width=0.45\textwidth]{resp090617a} & \includegraphics[width=0.45\textwidth]{resp090617b}\tabularnewline
\hline 
\includegraphics[width=0.45\textwidth]{/Users/rpolkinghorne/Desktop/oscor/resp160617a} & \includegraphics[width=0.45\textwidth]{/Users/rpolkinghorne/Desktop/oscor/resp160617b}\tabularnewline
\hline 
\includegraphics[width=0.45\textwidth]{resp090617c} & \includegraphics[width=0.45\textwidth]{resp090617d}\tabularnewline
\hline 
 & \tabularnewline
\hline 
\end{tabular}
\end{figure}
The quartic oscillator limit

Presumably a Bogoliubov state reduces to a squeezed state in the small
cloud limit where there is only one mode? As expected, $g^{2}(0)$
is conserved in a quartic oscillator. The Bogoliubov quadratures are
qualitatively similar to the coherent ones.
\begin{figure}

\caption{The small gas limit}
\begin{tabular}{|c|c|}
\hline 
\includegraphics[width=0.45\textwidth]{resp090617i} & \includegraphics[width=0.45\textwidth]{resp090617j}\tabularnewline
\hline 
\includegraphics[width=0.45\textwidth]{resp090617k} & \includegraphics[width=0.45\textwidth]{resp090617l}\tabularnewline
\hline 
\end{tabular}

\end{figure}
 

\section*{Derivation of Bogoliubov coefficients}

In the Hamiltonian of Equation\textasciitilde{}\ref{eq:ii}, let $\hat{H}=\int\hat{H}(z)\,dz$,
and substitute the mean-field form $\hat{\psi}(z)=\psi_{0}+\delta\hat{\psi}(z)$.
Expanding to second order in $\delta\hat{\psi}$ gives \textbf{check
the multinomial coefficients}{*}
\[
\begin{aligned}\hat{H}(z) & =\hat{\psi}{}^{\dagger}(x)\left(-\partial_{z}^{2}-1\right)\hat{\psi}(x)+\sqrt{\frac{\gamma}{2}}\left(\hat{\psi}^{\dagger}(x)\right)^{2}\hat{\psi}^{2}(x)\\
 & =\psi_{0}^{*}\left(-\partial_{z}^{2}-1\right)\psi_{0}+\sqrt{\frac{\gamma}{2}}\left|\psi_{0}\right|^{4}\\
 & \qquad+\delta\hat{\psi}^{\dagger}\left(-\partial_{z}^{2}+\sqrt{2\gamma}\left|\psi_{0}\right|^{2}-1\right)\psi_{0}+\text{h.c.}\\
 & \qquad\delta\hat{\psi}^{\dagger}\left(-\partial_{z}^{2}+2\sqrt{2\gamma}\left|\psi_{0}\right|^{2}-1\right)\delta\hat{\psi}+\sqrt{\frac{\gamma}{2}}\left(\psi_{0}^{*}\right)^{2}\delta\hat{\psi}^{2}+\sqrt{\frac{\gamma}{2}}\psi_{0}^{2}\left(\delta\hat{\psi}^{\dagger}\right)^{2}\\
 & \qquad+O\left(\delta\hat{\psi}^{3}\right).
\end{aligned}
\]
With a uniform and real order parameter$\psi_{0}^{2}=1/\sqrt{2\gamma}$,
the first line is a constant $1/2\sqrt{2\gamma}$, the second line
is zero, and the third line becomes
\[
\hat{H}(x)=\left(-\partial_{z}^{2}+1\right)\delta\hat{\psi}^{\dagger}\delta\hat{\psi}+\frac{1}{2}\delta\hat{\psi}^{2}+\frac{1}{2}\left(\delta\hat{\psi}^{\dagger}\right)^{2}.
\]
In this case, performing a Bogoliubov transformation
\[
\delta\hat{\psi}(x)=\sqrt{\frac{2\xi}{L}}\sum_{n=1}^{\infty}\left(\left(u_{n}^{*}a_{n}-v_{n}a_{n}^{\dagger}\right)\sin(\kappa_{n}x)+\left(u_{n}^{*}b_{n}-v_{n}b_{n}^{\dagger}\right)\cos(\kappa_{n}x)\right)
\]
where $\kappa_{n}=k_{n}\xi=2\pi n\xi/L$ is the wave number in terms
of the healing length, and each term is an eigenfunction of the Laplacian
belonging to the eigenvalue $-\kappa_{n}^{2}$, gives $H=1/2\sqrt{2\gamma}+\sum_{n}H_{an}+H_{bn}$,
where \textbf{(The constant is the energy difference between the coherent
state (which we set to zero) and the correlated Bogoliubov state.
The Bogoliubov state should have lower energy, so this should be negative.)}
\[
\begin{aligned}H_{an} & =\left(\kappa_{n}^{2}+1\right)\left(\left|v_{n}\right|^{2}+\left(\left|u_{n}\right|^{2}+\left|v_{n}\right|^{2}\right)a_{n}^{\dagger}a_{n}-u_{n}^{*}v_{n}^{*}a_{n}^{2}-u_{n}v_{n}\left(a_{n}^{\dagger}\right)^{2}\right)\\
 & \qquad+\frac{1}{2}\left(-u_{n}^{*}v_{n}-2u_{n}^{*}v_{n}a_{n}^{\dagger}a_{n}+\left(u_{n}^{*}\right)^{2}a_{n}^{2}+v_{n}^{2}\left(a_{n}^{\dagger}\right)^{2}+\text{h.c.}\right)
\end{aligned}
\]
and $H_{bn}$is formed from $H_{an}$by substituting $b_{n}$for $a_{n}$.
In order for $\delta\hat{\psi}(x)$ to satisfy the canonical commutation
relation for a Bose field operator, the coefficients must satisfy
$u_{n}^{2}-v_{n}^{2}=1$. To cancel the terms in $\left(a_{n}^{\dagger}\right)^{2}$
and diagonalise the Hamiltonian, we need 
\[
u_{n}^{2}+v_{n}^{2}-2\left(\kappa_{n}^{2}+1\right)u_{n}v_{n}=0,
\]

with solution
\[
u_{n}^{2}=\cosh^{2}r=\frac{\kappa+\kappa^{-1}}{2\sqrt{\kappa^{2}+2}}+\frac{1}{2}\qquad v_{n}^{2}=\sinh^{2}r=\frac{\kappa+\kappa^{-1}}{2\sqrt{\kappa^{2}+2}}-\frac{1}{2},
\]
where 
\[
2r=\tanh^{-1}\frac{1}{\kappa_{n}^{2}+1}
\]
has the form of a squeezing parameter. The large $\kappa$ limit is
$v^{2}\approx1/4\kappa^{4}$. The small $\kappa$ limit is $v^{2}\approx1/2\sqrt{2}\kappa$,
whence the density of uncondensed atoms in the longest wavelength
mode is $1/4\pi\sqrt{2}$.

The Hamiltonian reduces to
\[
\begin{aligned}H_{an} & =\left(\kappa_{n}^{2}+1\right)\left(\frac{\kappa+\kappa^{-1}}{2\sqrt{\kappa^{2}+2}}-\frac{1}{2}\right)-\sqrt{\frac{\left(\kappa+\kappa^{-1}\right)^{2}}{4\left(\kappa^{2}+2\right)}-\frac{1}{4}}+a_{n}^{\dagger}a_{n}\left(\left(\kappa_{n}^{2}+1\right)\frac{\kappa+\kappa^{-1}}{\sqrt{\kappa^{2}+2}}-2\sqrt{\frac{\left(\kappa+\kappa^{-1}\right)^{2}}{4\left(\kappa^{2}+2\right)}-\frac{1}{4}}\right)\\
 & =\frac{1}{\sqrt{\kappa^{2}+2}}\left(\frac{1}{2}\left(\kappa_{n}^{2}+1\right)\left(\kappa+\kappa^{-1}-\sqrt{\kappa^{2}+2}\right)-\frac{1}{2\kappa}+a_{n}^{\dagger}a_{n}\left(\left(\kappa_{n}^{2}+1\right)\left(\kappa+\kappa^{-1}\right)-\frac{1}{\kappa}\right)\right)\\
 & =\cdots+\kappa_{n}\sqrt{\kappa_{n}^{2}+2}a_{n}^{\dagger}a_{n}
\end{aligned}
\]

The total particle number is
\[
\begin{aligned}\left\langle n\right\rangle =\left|\psi_{0}\right|^{2} & +\left\langle \delta\hat{\psi}^{\dagger}\delta\hat{\psi}\right\rangle \\
 & +\frac{2\xi}{L}\sum_{n=1}^{\infty}\left|v_{n}\right|^{2}\\
 & +\frac{2\xi}{L}\sum_{n=1}^{\infty}\frac{\kappa+\kappa^{-1}}{2\sqrt{\kappa^{2}+2}}-\frac{1}{2}\\
 & +\frac{1}{\pi}\int_{0}^{\infty}\frac{\kappa+\kappa^{-1}}{2\sqrt{\kappa^{2}+2}}-\frac{1}{2}\,d\kappa
\end{aligned}
\]
where the last line takes the $L\gg\xi$ limit of infinite extent.
(The integral diverges, as expected in 1D where the condensate fraction
is zero.) In the units used here, the density of uncondensed bosons
is independent of $\gamma$, but the condensate density is $1/\sqrt{2\gamma}$
. 

The $\kappa=0$ condensate mode, with $u_{n}=1$ and $v_{n}=0$, gives
\[
H_{a0}=\left(a_{n}^{\dagger}a_{n}\right)+\frac{1}{2}a_{n}^{2}+\frac{1}{2}\left(\text{\ensuremath{a_{n}^{\dagger}}}\right)^{2}.
\]
The mean field approximation, with a coherent condensate mode, is
inconsistent to second order. The consistent solution requires infinite
squeezing, in which case $\delta\hat{\psi}$ is no longer small. So
maybe the Bogoliubov state should not be expected to give sensible
results for quadratures or other parameters of the condensate orbital.

\bibliographystyle{plain}
\bibliography{pra-0-0}

\end{document}
